% !TeX encoding = UTF-8
% !TeX root = PySABr.tex
% !TeX spellcheck = en_US

% Preamble
% ---
\documentclass{article}


% Packages
% ---
\usepackage{amsmath}
\usepackage{hyperref}
\usepackage{graphicx}
\usepackage{listings}

% \usepackage[backend=bibtex,style=verbose-trad2]{biblatex}
% \bibliography{EloXL}


% Opening
% ---
\title{Elo Ratings in Football}
\author{Alex Munger}


% Main document
% ---

\begin{document}

\maketitle

\begin{abstract}
	This is the abstract. Elo!
\end{abstract}

\section{Background}

Elo ratings take 2 rating inputs for a match-up and an input result and calculate new Elo ratings for the teams in question.

In this implementation, each team receives an initial Elo rating of $ 1500 $. The possible results are $ \in \left\lbrace 0, 0.5, 1 \right\rbrace $ representing a loss, draw, and win, respectively.

\section{Equations}

\begin{align*}
	Q_{a} &= 10^{R_{a} / 400} \\
	Q_{b} &= 10^{R_{b} / 400} \\
	E{a} &= \dfrac{Q_{a}}{Q_{a} + Q_{b}} \\
	E{b} &= \dfrac{Q_{b}}{Q_{a} + Q_{b}} \\
\end{align*}

\end{document}
